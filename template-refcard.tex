%% template-refcard.tex
%% Copyright 2014, 2015  Rudy Matela
%
% This work may be distributed and/or modified under the
% conditions of the LaTeX Project Public License, either version 1.3
% of this license or (at your option) any later version.
% The latest version of this license is in
%   http://www.latex-project.org/lppl.txt
% and version 1.3 or later is part of all distributions of LaTeX
% version 2005/12/01 or later.
%
% The Current Maintainer of this work is Rudy Matela.
%
% This work consists of the files refcard.cls and tempalte-refcard.tex


\documentclass{refcard}
\usepackage[T1]{fontenc} % necessary so '{', '}' and '\' get the right font

\renewcommand{\familydefault}{\sfdefault}

\title{Template\\Reference Card / Cheat Sheet}

\cright{
	Copyright \textit{0000}  \textit{Author Name}
}{
	Avaliable under \textit{License Name} \tiny This is a template, see source for real license
}
\version{0.4.0}


\begin{document}

\maketitle

\section{Structure}

\begin{verbatim}
	Something {
	    Command
	    Statement
	    Command
	}
\end{verbatim}


\section{Types}

\begin{ldesc}
	\li[char]    char
	\li[integer] integer
	\li[list]    list
	\li[pointer] *char, *integer, ...
\end{ldesc}


\section{Initialization}

\begin{ldesc}
	\li[local variable]     \I{type} \I{name}=\I{value}
	\li[global variable]    \I{type} \I{NAME}=\I{value}
	\li[volatile variable]  \I{type} \I{\_name\_}=\I{value}
	\li[lists]              \I{type} \I{name}[\I{size}]=\I{value}
\end{ldesc}


\section{Literals}

\begin{ldesc}
	\li[String with escapes]    "str..$\backslash{}$a.ing"
	\li[String without escapes] 'str..$\backslash{}$a.ing'
	\li[Character]              \#x
	\li[Character (Unicode)]    \#\{12AB56\}
\end{ldesc}


\section{Operators (grouped by precedence)}

\begin{Ldesc}
	\Li[object member                ] \I{name}.\I{member}
	\li[module member                ] \I{name}::\I{member}
	\Li[plus, minus (unary), negation] +, -, ~
	\Li[multiplication, division, remainder] *, /, \%
	\li[division\&remainder          ] \I{qo},\I{rt} <- \I{dd} /\% \I{ds}
	\Li[sum, subtraction             ] +, -
	\Li[comparisons                  ] >, >=, <, <=
	\Li[comparisons                  ] ==, !=
	\Li[and, or                      ] \&\&, ||
	\Li[ternary conditional          ] \I{expr1}?\I{expr2}:\I{expr3}
	\Li[assignments                  ] <-, +<-, -<-, *<-
	\li                                /<-, ||<-, \&\&<-
\end{Ldesc}

\noindent
Unary operators and conditionals group right to left.\\
All other group left to right.


\section{Control Flow}

\begin{ldesc}
	\li[statement terminator]        ;
	\li[block delimiter]             \{ \} or [ ]
	\li[exit from while]             break
	\li[next while iteration]        continue
	\li[return value from function]  return \I{expr}
\end{ldesc}


\subsection{Flow Constructions}

\begin{ldesc}
	\li[if statement]
		if~(\I{expr})~\I{block} \li
		else~~~~~     \I{block} \li

	\li[while statement]
		while (\I{expr}) \{ \li
		~~~~\I{statement};  \li
		~~~~\I{statement};  \li
	    \}
\end{ldesc}


\section{Libraries}


\subsection{Lists \hfill \C{import Libs.StdList}}

\begin{ldesc}
	\li[iiii] iiii
	\li[wwww] wwww
	\li[iiii] wwww
	\li       wwww
	\li[blah] blah
\end{ldesc}

\subsection{Mathematical Functions \hfill \C{inc Some.Math}}

\begin{ldesc}
	\li[iiii] iiii
	\li[wwww] wwww
	\li[iiii] wwww
	\li       wwww
	\li[blah] blah
\end{ldesc}


\subsection{Input and Output \hfill \C{require System.IO}}

\begin{ldesc}
	\li[printing] print "formatstring"
	\li[wwww] wwww
	\li[iiii] wwww
	\li       wwww
	\li[blah] blah
\end{ldesc}

% Debug for three column environment:
%\newpage \subsection{Blah} Bleh  11111
%\newpage \subsection{Blah} bleh  22222
%\newpage \subsection{Blah} bleh  33333


\end{document} 

