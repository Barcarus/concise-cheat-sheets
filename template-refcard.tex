\documentclass{refcard}

\renewcommand{\familydefault}{\sfdefault}

\title{Somelanguage\\Reference Card / Cheat Sheet}

\begin{document}

\maketitle

\section{Structure}

\begin{verbatim}
	Such language {
	    Much beautiful;
	    Clean and consistent,
	    Want syntax |
	]
\end{verbatim}


\section{Types}

\begin{ldesc}
	\lditem{char   }{char                }
	\lditem{integer}{integer             }
	\lditem{list   }{list                }
	\lditem{pointer}{*char, *integer, ...}
\end{ldesc}


\section{Initialization}

\begin{ldesc}
	\lditem{local variable   }{ such \I{type} \I{name}=\I{value}           }
	\lditem{global variable  }{ much \I{type} \I{name}=\I{value}           }
	\lditem{volatile variable}{ so   \I{type} \I{name}=\I{value}           }
	\lditem{lists            }{ such \I{type} \I{name}[\I{size}]=\I{value} }
\end{ldesc}


\section{Literals}

\begin{ldesc}
	\lditem{String with escapes   }{"str..$\backslash{}$a.ing"}
	\lditem{String without escapes}{'str..$\backslash{}$a.ing'}
	\lditem{Character             }{\#x}
	\lditem{Character (Unicode)   }{\#\{12AB56\}}
\end{ldesc}


\section{Operators (grouped by precedence)}

\begin{ldesc}
	\hline
	\lditem{object member                }{\I{name}.\I{member}}
	\lditem{module member                }{\I{name}::\I{member}}
	\hline
	\lditem{plus, minus (unary), negation}{+, -, ~}
	\hline
	\lditem{multiplication, division, remainder}{*, /, \%}
	\lditem{division\&remainder          }{\I{qo},\I{rt} <- \I{dd} /\% \I{ds}}
	\hline
	\lditem{sum, subtraction             }{+, -}
	\hline
	\lditem{comparisons                  }{>, >=, <, <=}
	\hline
	\lditem{comparisons                  }{==, !=}
	\hline
	\lditem{and, or                      }{\&\&, ||}
	\hline
	\lditem{ternary conditional          }{\I{expr1}?\I{expr2}:\I{expr3}}
	\hline
	\lditem{assignments                  }{<-, +<-, -<-, *<-}
	\lditem{                             }{/<-, ||<-, \&\&<-}
	\hline
\end{ldesc}

\noindent
Unary operators and conditionals group right to left.\\
All other group left to right.


\section{Control Flow}

\begin{ldesc}
	\lditem{statement terminator      }{;}
	\lditem{block delimiter           }{\{ ] or [ \}}
	\lditem{exit from while           }{brk}
	\lditem{next while iteration      }{cnt}
	\lditem{return value from function}{return \I{expr}}
\end{ldesc}


\subsection{Flow Constructions}

\begin{ldesc}
	if statement    & if\s(\I{expr})\s\I{block} \\
			        & else\s\s\s\s\s\s\s\s\I{block} \\
				    & \\
	while statement & while (\I{expr}) \{ \\
	                & \s\s\s\s\I{statement}; \\
			        & \s\s\s\s\I{statement}; \\
			        & \} \\
\end{ldesc}


\section{Libraries}


\subsection{Lists \hfill \C{import Such.StdList}}


\subsection{Mathematical Functions \hfill \C{inc So.Math}}


\subsection{Input and Output \hfill \C{require Much.IO}}


The quick brown fox jumps over the lazy dog.
The quick brown fox jumps over the lazy dog.
The quick brown fox jumps over the lazy dog.
The quick brown fox jumps over the lazy dog.
The quick brown fox jumps over the lazy dog.
The quick brown fox jumps over the lazy dog.
The quick brown fox jumps over the lazy dog.
The quick brown fox jumps over the lazy dog.

\section{Blah}

The quick brown fox jumps over the lazy dog.
The quick brown fox jumps over the lazy dog.
The quick brown fox jumps over the lazy dog.
The quick brown fox jumps over the lazy dog.
The quick brown fox jumps over the lazy dog.
The quick brown fox jumps over the lazy dog.
The quick brown fox jumps over the lazy dog.
The quick brown fox jumps over the lazy dog.

\end{document} 

